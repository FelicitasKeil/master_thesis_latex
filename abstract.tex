Apart from single gravitational wave (GW) events, those with a smaller amplitude produce a stochastic GW background. This has not been measured yet using traditional GW detectors, like the Laser Interferometer Gravitational-Wave Observatory (LIGO), Virgo or the Kamioka Gravitational Wave Detector (KAGRA). With future detectors, such as Einstein Telescope (ET) or Cosmic Explorer (CE), it is expected to be observed in the frequency range of $\approx 10-10^4$ Hz.  The astrophysical gravitational wave background (AGWB) is modelled here in a frequency-dependent way using the {\tt Multi\_CLASS} code. We assume the design noise of a cross-correlation between ET and CE at the dipole and use this as the variance of a Gaussian noise profile. Then, we use Information Field Theory (IFT) to separate our calculated power spectra from this noise. This returns some possible reconstruction at a high AGWB power spectrum at 400 Hertz but does not work for a lower power spectrum at 100 Hertz. We also use this method with the cosmological background anisotropies from {\tt CLASS\_GWB}, where the noise is comparatively too high for a successful reconstruction.
