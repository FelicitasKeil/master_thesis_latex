\documentclass[main.tex]{subfiles}

\textit{finish at the end}

Many GW have an amplitude below the necessary signal-to-noise ratio to be detected individually. These sources form the GW background. The first detection of this background was made in 2023 by the NANOGrav pulsar time array with an energy density parameter of $\Omega_{GW} = 9.3^{+5.8}_{-4.0}\cdot 10^{-9}$ \cite{agazie_nanograv_2023}. There is a predicted AGWB and a predicted cosmological GW background.
In current GW experiments, like LIGO, Virgo and KAGRA, the noise is too high to detect this background. Experiments with a higher sensitivity are planned for the future, like the ground-based Einstein Telescope (ET) and Cosmic Explorer (CE) or the space-based Laser Interferometer Space Antenna (LISA) and TianQin. These might have low enough noise to detect the GW background and even disentangle different components.

With new detectors in the future we could measure intrinsic anisotropies in this background which are not coming from our observer motion or statistical properties, so-called shot noise from a Poisson distribution. These anisotropies, like for example the dipole, would contradict the cosmological principle. It stated that the universe is homogeneous and isotropic at every point.
