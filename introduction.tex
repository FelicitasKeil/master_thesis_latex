\documentclass[main.tex]{subfiles}

Most large-scale structure surveys, like Planck [\cite{planck_collaboration_planck_2016}], have found an unexpectedly large dipole anisotropy, which can be used to test the cosmological principle. The principle states that the universe is isotropic and homogeneous on large scales. These surveys have error sources, such as partial sky covering, due to masking sources and the fact that clustering of matter can produce an anisotropic contribution. GW have a very high sky coverage since masking is not necessary for GW detectors. Thus, we can use GW observations to test the cosmological principle in an independent way. This is possible because both tracers are related to the underlying matter distribution that was seeded in the early universe. 

To do this, we can look at the AGWB. Many GW have an amplitude below the necessary signal-to-noise ratio to be detected individually. These sources form the GW background.  Compact binary mergers are expected to make up the majority of this background in our considered frequency range of 1-1000 Hz.
This background has been detected in a low-frequency range of $10^{-8.75}$ - $10^{-7.5}$ Hz in 2023 by the North American Nanohertz Observatory for Gravitational Waves (NANOGrav) pulsar time array with an energy density parameter of $\Omega_{GW} = 9.3^{+5.8}_{-4.0}\cdot 10^{-9}$ [\cite{agazie_nanograv_2023}]. 
In current GW experiments, like LIGO, Virgo and KAGRA (LVK), the noise is too high to detect this background. Experiments with a higher sensitivity are planned for the future, like the ground-based ET and CE. These might have low enough noise to detect the GW background and even disentangle different components. With these future detectors, we could possibly measure intrinsic anisotropies in this background which are not coming from our observer motion or statistical properties, i.e. shot noise from a Poisson distribution. A larger-than-expected intrinsic dipole or higher multipole would then contradict the cosmological principle. 

We compute the anisotropies of the AGWB in a frequency-dependent way using a modified version of {\tt Multi\_CLASS} [\cite{bellomo_beware_2020}] which is based on {\tt CLASS} [\cite{blas_cosmic_2011}], an Einstein-Boltzmann solver. The used code is publicly available \footnote{\url{https://github.com/FelicitasKeil/Multi_CLASS}}. It makes sense to use an Einstein-Boltzmann solver since the formalism of computing the AGWB angular power spectrum is similar to doing this for the cosmic microwave background. 

In the next step, we model the noise using design sensitivities of ET+CE and use a separation method from IFT on the AGWB at different frequencies, as well as on the cosmological GW background. Separating the intrinsic anisotropies of the background from noise would give us insight into the astrophysical parameters that govern this background and into the validity of the cosmological principle. 

This thesis is structured as follows: In Chapter \ref{gw_chapter}, we discuss the fundamental physics of GW and their stochastic background. Then, the frequency dependence of the AGWB is explained in Chapter \ref{frequency_chapter}. The instrumental noise of current and future experiments is presented in Chapter \ref{comp_sep}. IFT and the used techniques are summarised in Chapter \ref{ift_chapter}. Then, we present the results in Chapter \ref{results_chapter} before concluding and giving an outlook.