The goal of this work was to compute frequency-dependent AGWB anisotropies and forecast whether or not a network of next-generation ground-based GW detectors could measure the AGWB. We used the resulting angular power spectra in an IFT separation with respect to a Gaussian noise at ET+CE levels.

To do that, we considered the stochastic GW background of BBH using the number density distribution from {\tt Multi\_CLASS}. For frequencies up to around $ 1000$ Hz, mergers containing neutron stars are negligible compared to BBH, which is why we only consider BBH in this formalism.
The {\tt Multi\_CLASS} code was modified to compute the frequency-dependent angular power spectra of the AGWB. The frequency dependence comes from the window function which weights projection effects at different redshifts. These projection effects stem from the line of sight integration and include redshift space distortions, density fluctuations, the Doppler and further general relativity effects. 

The window function depends on the BBH merger rate and the energy spectrum of one binary system. We consider all three phases of binary coalescence, i.e. inspiral, merger and ringdown. For the merger rate, we model the SFR since BH come from stellar evolution. We also parametrise the HMF to arrive at the merger rate. A further frequency dependence comes from the evolution bias. However, this is negligible in the final power spectrum. For the noise, we use design sensitivities of ET cross-correlated with CE using scale-independent Gaussian noise as an approximation.

Afterwards, we use the resulting anisotropies for an IFT separation with the {\tt NIFTy} package. The separation does not work for a lower angular power spectrum at 100 Hz but works roughly for a higher one at 400 Hz.
To test the same method with the cosmological GW background, we use {\tt GW\_CLASS} assuming an inflationary background at 900 Hz. Since this background is below the astrophysical one, the reconstruction is not possible.

Apart from the instrumental noise, the kinematic dipole and statistical shot noise also obscure the intrinsic AGWB. In the future, those can be modelled using the same formalism and can then be included in the IFT separation. The instrumental noise can also be modelled in a scale-dependent way to be more realistic, i.e. increasing at higher multipoles. 

Furthermore, astrophysical parameters, like the mass of each merging BH and the inclination angle, could be drawn from a physical probability distribution. The framework can also be extended to consider neutron star mergers.

Computing one angular power spectrum at a frequency is still computationally costly at the moment. Improving the speed of the implementation would also allow computing more frequencies to see the full frequency dependence. It would then be interesting to use the functional frequency dependence directly in {\tt NIFTy}. This should improve the separation since it is an extra dimension that can be used in IFT. Using more powerful IFT methods other than the Wiener filter would help extract precise information from an eventual data set as well. The next step would then be separating the AGWB from the cosmological one, either without or in the presence of instrumental noise.

With a well-modeled AGWB we can hope to have a detection using next or next-to-next generation GW detectors. This would tell us more about the properties and sources of this background and shine light on the cosmological principle.