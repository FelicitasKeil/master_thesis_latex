We considered the stochastic GW background of BBH using the number density distribution from {\tt Multi\_CLASS}. GW waves can help to test the cosmological principle since they have better sky coverage than other large scale structure surveys. For frequencies up to $\approx 1000$ Hertz, mergers containing neutron stars are negligible compared to BBH, which is why we only consider BBH in this formalism.
The code was then modified to compute the frequency-dependent angular power spectra of the AGWB. The frequency dependence comes from the window function which weights projection effects at different redshifts. These projection effects come from the line of sight integration and include redshift space distortions, density fluctuations, the Doppler and further general relativity effects. 

The window function depends on the BBH merger rate and the energy spectrum of one GW. We consider all three phases of binary coalescence, i.e. inspiral, merger and ringdown. For the merger rate, we model the SFR since BH come from stellar evolution. We also parametrise the HMF because stars form in haloes. Then, we can integrate over the halo mass to arrive at the merger rate. 
A further frequency dependence comes from the evolution bias, also calculated from the merger rate and the energy spectrum. However, this is negligible in the final power spectrum.
We model the noise from design sensitivities of ET+CE using scale-independent Gaussian noise as an approximation.

Afterwards, we use the resulting anisotropies for an IFT separation with the {\tt NIFTy} package. The separation does not work at a lower angular power spectrum at 100 Hertz but works roughly at a higher one at 400 Hertz.
To test this with the cosmological GW background, we use {\tt GW\_CLASS} assuming an inflationary background at 900 Hertz. Since this background is below the astrophysical one, the reconstruction is not possible.

In the future, it would be interesting to extend this and use the frequency dependence directly in {\tt NIFTy}. This should improve the separation since it is an extra dimension that can be used to perform the separation. The noise can also be modelled in a scale-dependent way to be more realistic. Computing one angular power spectrum at a frequency is still computationally costly at the moment. Improving the speed of the implementation would also allow computing more frequencies to see the full frequency dependence.
The next step would be to separate the AGWB from the cosmological one either without or in the presence of instrumental noise.